\begin{center}
\huge QuickStart Guide\\[5.5mm]
\end{center}
Diese Anleitung beschreibt wie man in 7 Schritten ein eigenes
OpenVPN-Netzwerk mit dem \Nbm{} erstellen kann.\\[2mm]

\colhead{Start}
\picContentRight{config}{
Entpacke den \Nbm{} und starte ihn mit einem Klick auf die
``start-manager''-Datei im entpackten Ordner. Wird der \Nbm{} zum ersten Mal
gestartet, erscheint ein Basis-Konfigurationsdialog.
}{width=85mm,height=60mm}

\colheadRight{Erzeugung der Benutzer}
\picContentLeft{user1}{
Der
n"achste Schritt ist das Erzeugen der Benutzer. Mit einem Klick auf
``Bearbeiten-$>$Neuer Benutzer anlegen'' kann ein neuer Benutzer angelegt
werden.
}{width=75mm,height=52mm}

\colhead{Erzeugung des Servers}
\picContentRight{server1}{
Klicke auf ``Bearbeiten-$>$Neuen Server anlegen'' um einen neuen Server
anzulegen. Haben Sie einen boa-server, so k"onnen sie dies dem \Nbm{} unter
\textit{Server Typ} im ersten Tab mitteilen, woraufhin dieser ihnen die restliche
Konfiguration des Servers abnimmt.
}{width=76mm,height=56mm}

\newpage

\colheadRight{Syncing}
\picContentLeft{sync}{
Klicke mit Rechts auf den Server und w"ahle ``\textit{Synchronisiere
Benutzer}''  um mit dem Server zu
synchronisieren. Hierbei verbindet sich der \Nbm{} mit dem Server und l"adt
alle, f"ur den Server relaventen Dateien, auf ihn.
}{width=76mm,height=56mm}

\colhead{Server konfigurieren}
\picContentRight{startVpn}{
Da der \Nbm{} automatisch eine Server-Config erstellt, gen"ugt es diese auf
den Server zu laden und danach den OpenVPN-Prozess zu starten.\\
Haben Sie eine boa so l"adt der \Nbm{} die Server-Config w"ahrend der
Synchronisation selbstst"andig hoch.
}{width=76mm,height=56mm}

\colheadRight{Verbinden und Testen}
\picContentLeft{control}{
Der \Nbm{} erstellt automatisch OpenVPN-Userconfigs und speichert sie im
Export-Verzeichnis unter den jeweiligen Benutzer.\\
Ein weiteres Feature ist das Kontrollzentrum des \Nbm{}. Wenn es im
Konfigurationsdialog aktiviert wurde, so gen"ugt es den gew"unschten Server
zu selektieren (nicht "offnen) und man kann im Hauptpanel den
Kontrollzentrum-Button dr"ucken.\\
Das Kontrollzentrum kann nur verwendet werden wenn ein OpenVPN-Prozess auf
dem Server l"auft. Es erm"oglicht eigene Kommandos an den Server zu senden
sowie die verbunden Benutzer zu "uberwachen.
}{width=76mm,height=56mm}

\colhead{War das alles ?}
Nein, der Quickstart Guide ist nur eine kleine Einf"uhrung in den \Nbm{}, welcher
viele Funktionen nicht aussprechend ber"ucksichtigt.\\
Deswegen sollten Sie auf jeden Fall das \Nbm{}-Handbuch lesen, in welchen
alle Funktionen des \Nbm{} ausf"uhrlich beschrieben werden.
