\section{Installation}
\newitem{Installieren \& Starten}

%
Die \Nvma{} ist zu entpacken. Dabei ist \Nbn{} die Versions-Nummer bzw. die
Build-Nummer bei Entwicklerversionen.
%
\nl Das Archiv enth"alt folgenden Inhalt:
%
\begin{items}
 \item	\Nvmj{}: Hauptanwendung als JAR Archiv.
 \item	\Nvmw{}: dies ist die ausf"uhrbare Datei f"ur Windows und Mac OS X
 \item	\Nvms{}: dies ist das ausf"uhrbare JAR Archiv.
 \item	\code{db/}: in diesem Verzeichnis wird die Datenbank des
	Programms gespeichert. Aus diesem Grund sollte die
	Installation auf einem verschl"usseltem Datentr"ager
	aufbewahrt werden.
 \item	\code{lib/}: externe Programmbibliotheken.
\end{items}
%
Zum Ausf"uhren der Anwendung ist ein Java Runtime Environment
der Version 6 oder 7 notwendig.
%
Auf den Plattformen Microsoft Windows und Mac OS X (ab Version 10.5.x)
kann die Applikation "uber einen Doppelklick auf die \Nvmw{} bzw. \Nvms{}
gestartet werden.
%
"Uber die Kommandozeile auf allen Plattformen geht dies mittels:

\nl{\small\texttt{\ \ \ java -jar\ }\Nvms{}}\\

\newitem{Erste Anwendungsschritte}
%
Die Anwendung startet mit einem Konfigurationsdialog. Die hier getroffenen
Einstellungen bilden die Basis zur Arbeit mit dem \Nbm.
Beispielsweise kann man die Module, welche in der
Anwendung aktiv sein sollen, w"ahlen.

\nl Folgende Module stehen zur Auswahl:

\begin{items}
	\item Zertifikatsverwaltung (dies beinhaltet auch das Benutzermanagement)
	\item Leitstand: eine Verwaltungsschnittstelle zur Kontrolle laufender OpenVPN-Server.
\end{items}

\nl Zus"atzlich kann in der Konfiguration die Kodierung der Zertifikate
(\code{PKCS\#12} oder \code{Base64}) sowie ein Exportverzeichnis angegeben werden,
in welchem erzeugte oder importierte Zertifikate und Schl"ussel gespeichert werden.

%
Weiterhin kann festgelegt werden, ob zu Konzentratoren und Benutzern
passende Konfigurationsdateien zur Verwendung mit OpenVPN erzeugt werden sollen.
Diese werden bei Aktivierung der Funktion bei jeder "Anderung an einem Server
oder Benutzer automatisch generiert und im zugeh"origen Verzeichnis im 
Dateisystem gespeichert.


\newitem{Import bestehender Daten}

\nl Sollte bereits eine Datenbank f"ur den \Nbm{} vorhanden sein,
kann diese im
Konfigurationsdialog angegeben werden. Sollte man seine Datenbank wechseln,
so ist ein Neustart des \Nbm{} erfordlich. Noch vor dem Neustart gemachte
"Anderungen werden nicht "ubernommen, beziehungsweise zugelassen.\\
W"ahrend des Dialoges wird
desweiteren gefragt, ob ein Import bestehender Daten, z.B.
 Root-CAs, bestehende Benutzerzertifikate, etc., stattfinden soll.
Der Importer
kann auch jederzeit "uber den Men"ueintrag ``Datei$\,\rightarrow\,$Benutzer Import''
erneut angestossen werden.

\nl Es ist m"oglich sowohl aus dem Dateisystem, als auch aus einer
LDAP-Verzeichnisstruktur heraus, zu importieren.
%
Anschlie"send kann der Bediener entscheiden, ob aus den CNs\footnote{%
\sffamily ``Common Name'' -- spezifischer Textteil eines Zertifikats}
der Subjects der Zertifikate Benutzer erzeugt werden sollen.
Diese werden dann automatisch miteinander verkn"upft.


\nl Wurde sich gegen einen Import entschieden, besteht die M"oglichkeit, die
Basiskonfiguration f"ur die sogenannte \emph{Certificate Authority} zu erstellen.

\nl\textbf{Achtung: }\begin{quote}In manchen F"allen kann es aufgrund von
Exportbeschr"ankungen in der Java Umgebung vorkommen, dass eine h"ohere
Schl"usselst"arke als 1024 Bit zu Fehlern f"uhrt.  Sicherheitshalber
sollten Sie in diesem Fall vor dem Erzeugen von Zertifikaten neue Java
Policy Dateien in Ihr System einspielen.  \footnote{Sollten Sie von dem
Problem betroffen sein, finden Sie f"ur Java 6 und Java 7 die ``Java Cryptography Extension (JCE)
Unlimited Strength Jurisdiction Policy Files''  hier:
\url{http://www.oracle.com/technetwork/java/javase/downloads/index.html} Die
Dateien m"ussen anschlie"send nach JAVA\_HOME/lib/security kopiert werden,
wobei damit die bestehenden Policy Dateien "uberschrieben werden.}
\end{quote}

\newpage

\section{Verwaltung}

\newitem{Innerhalb der Anwendung}
%
Nach der Erstkonfiguration pr"asentiert sich die Anwendung, abh"angig von den
aktivierten Modulen, mit den folgenden Reitern:

\begin{items}
	\item X509: Der "Uberblick "uber die Zertifikatskonfiguration
	\item Server: Die Schnittstelle zur Verwaltung der einzelnen Server
	\item Benutzer: Die Schnittstelle zur Benutzerverwaltung
	\item Server/Benutzer: Eine alternative Darstellung der konfigurierten
Zugriffe auf die VPN-Server der einzelnen Benutzer
\end{items}
%
% currently not supported
%\subpoint{Updates}
%
%\nl "Uber ``Hilfe$\,\rightarrow\,$Update-Einstellungen'' k"onnen Zertifikate eingespielt werden,
%um Updates abzurufen. Diese Zertifikate werden Ihnen von \Nbtm{} zur Verf"ugung gestellt. Ohne
%g"ultige Zertifikate k"onnen keine Updates heruntergeladen werden.
%
%\nl Es k"onnen entweder eine einzelne pem-Datei, oder Zertifikat- und Key-
%Datei separat eingepflegt werden.
%
%\nl Weiterhin kann festgelegt werden, ob beim Anwendungsstart automatisch nach neuen
%Updates gesucht werden soll.
%
%\nl Die Lizenzdateien bekommen Sie von \Nbtm{}.
%
\subpoint{Import}

\nl "Uber ``Datei$\,\rightarrow\,$Benutzer Import'' kann der Import
(bei Bedarf auch mehrmals) angesto"sen werden. Es erscheint wiederum
ein Dialog, in welchem die Erzeugung von Benutzern aus den CNs
ausgew"ahlt werden kann.

\newitem{Verwaltung der Zertifikate}

\nl Die vom \Nbm{} verwalteten Zertifikate k"onnen jederzeit
"uber die ``X509''-Ansicht eingesehen werden. Hierzu kann man ein Zertifikat in der
"Ubersicht ausw"ahlen und durch Doppelklick "offnen. In dieser
Ansicht sind auch alle Details des Zertifikates sichtbar. Eine Option unter
 ``Bearbeiten$\rightarrow$Gesperrte Zertifikate anzeigen''
erm"oglicht es, sich nur die g"ultigen Zertifikate anzeigen zu lassen.

\nl Um das Zertifikat eines Benutzers noch schneller erreichen zu k"onnen, kann
auch der Benutzer ge"offnet werden (in der Ansicht ``Benutzer''). Im Detail-Dialog
des Benutzers kann "uber ``Anzeigen'' bei ``Zertifikat'' die Detail-Ansicht des
Zertifikats erreicht werden.

\subpoint{Root Zertifikate}

\nl Die Anwendung ist f"ur die Verwaltung von einem Root Zertifikat ausgelegt. Ausgehend davon
werden die Server- und Client-Zertifikate erstellt.

\nl Sollen mehrere Root Zertifikate mit dem \Nbm{} verwaltet werden kann dies durch das Anlegen
einer neuen Datenbank f"ur jedes zu verwaltende Root Zertifikat erreicht werden. Die Datenbanken k"onnen
einfach "uber den allgemeinen Konfigurationsdialog gewechselt werden.

\newitem{Serververwaltung}

\nl "Uber ``Bearbeiten$\,\rightarrow\,$Neuen Server anlegen'' k"onnen der
Anwendung Server hinzugef"ugt werden. Folgende Einstellungen sind im
Dialog m"oglich:

\begin{items}
	\item Bezeichnung: Hier kann ein beliebiger Name vergeben werden
	\item Hostname: Der Hostname oder die IP-Adresse des Servers
	\item Name im Zertifikat: Der Name (common name) f"ur das Zertifikat des Servers
	\item Abteilung: Die Abteilung (organizational unit) f"ur das Zertifikat des Servers
	\item Server-Typ: Hier kann ausgew"ahlt werden, ob es sich bei dem Server
	      um eine \Nboa{} oder um einen regul"aren OpenVPN
	      Server handelt. Soll der Manager mit einem regul"aren OpenVPN
	      Server verbunden werden, so beachten Sie bitte den Abschnitt
	      ``Verbindung mit einem regul"aren OpenVPN Server''.
	\item Art der Authentifizierung: Der  \Nbm{} kommuniziert mit dem
         Server "uber das SSH-Protokoll und kann sich entweder per
	      Passwort oder mittels eines SSH-Schl"ussels authentifizieren.
	\item Benutzername: Der Benutzername der f"ur die SSH-Verbindung
	      verwendet werden soll. Im Falle einer \Nboa{} ist dies der
	      Benutzer `manager'.
	\item Schl"usseldatei: Pfad zur ssh key Datei (nur notwendig falls
	      die Authentifizierung mittels eines Schl"ussels verwendet wird).
	\item Socket Befehl: Hier muss das Programm zur Socket-Kommunikation
	      inkl. vollst"andigem Pfad auf dem Server angegeben werden. Im
	      Falle einer \Nboa{} kann der Default-Wert "ubernommen werden.
	\item SSH Port: Port auf der Serverseite "uber den das SSH-Protokoll kommuniziert
	\item G"ultig: G"ultigkeit des Serverzertifikats
	\item Server-Pfad zur Passwd-Datei: Pfad auf der Serverseite, sofern
	      Dual-Factor Authentifizierung verwendet wird.
	\item Server-Pfad der Zertifikate: Pfad auf der Serverseite zum
	Verzeichnis, in welchem Root- und Serverzertifikat gespeichert werden
	sollen.
	\item OpenVPN Port: Der Port auf welchem der OpenVPN Server l"auft
	\item OpenVPN Protokoll: Das Protokoll welches der OpenVPN Server benutzt
	\item Netzwerkaddresse: Die Netzwerkaddresse welche der Server verwendet
	\item OpenVPN Device: Das \textit{device} (dt. Ger"at) welches der
	OpenVPN-D"amon verwenden soll.
	\item Redirect Gateway: Diese Option leitet den gesamten IP-Verkehr vom
	Benutzer "uber den Server.
	\item OpenVPN statische IP f"ur Benutzer: Diese Option erm"oglicht es den
	Benutzern, statische IP-Adressen zuzuweisen.
	\item Server-Pfad zum Konfigurationsverzeichnis: Pfad auf der Serverseite
	zum Konfiguratiosnverzeichnis der Benutzer. Dieses Feld ist notwendig
	wenn man statische IP-Adressen an Benutzer vergeben hat.


\end{items}

\nl "Uber das Pluszeichen k"onnen weitere Daten des Servers eingetragen werden:

\begin{items}
	\item Duplicate-CN: Erm"oglicht zwei Verbindungen mit dem gleichen
	Zertifikat, dh. ein Zertifikat kann von mehr als einem Benutzer verwendet
	werden.
	\item User: Nachdem der OpenVPN-D"amon gestartet wurde, l"auft er mit
	den Rechten von \textit{user} weiter.
	\item Group: Nachdem der OpenVPN-D"amon gestartet wurde, l"auft er mit
	den Rechten von \textit{group} weiter.
	\item Keep Alive: Das erste Feld gibt an, nach wievielen Sekunden der Server
	einen Ping senden soll, um zu "uberpr"ufen ob der Client noch erreichbar
	ist. Das zweite Feld gibt an, wieviele Sekunden der Server auf eine Antwort
	des Pings warten soll.
\end{items}


\nl Eine weitere M"oglichkeit Server anzulegen ist "uber den ``Server/Benutzer''-Reiter.
Ist dieser Reiter angew"ahlt, ist in der Toolbar eine Schaltfl"ache zum Anlegen von Servern
sichtbar.


\newitem{Benutzerverwaltung}

\nl "Uber ``Bearbeiten$\,\rightarrow\,$Neuen Benutzer anlegen'' k"onnen der
Datenbank Benutzer hinzugef"ugt werden. Folgende Einstellungen sind im
Dialog m"oglich:
\begin{items}

	\item Benutzername: Hier kann ein beliebiger Name f"ur den Benutzer vergeben werden
	\item Passwort: Das Passwort muss mindestens sechs (6) Zeichen lang sein
\end{items}

\nl "Uber das Pluszeichen k"onnen weitere Daten des Benutzers eingetragen werden:
\begin{items}
	\item Name im Zertifikat: Der Name (common name) f"ur das Zertifikat des Benutzers
	\item Abteilung: Die Abteilung (organizational unit) f"ur das Zertifikat des Benutzers
\end{items}

\nl Sind bereits Server im \Nbm{} angelegt worden, kann der Benutzer
bereits beim Anlegen diesen direkt zugewiesen werden.

\nl Es ist auch m"oglich "uber den ``Server/Benutzer''-Reiter neue Benutzer
anzulegen. Wenn dieser Reiter angew"ahlt ist, so erscheint in der Toolbar
eine Schaltfl"ache zum Anlegen von Benutzern.

\subpoint{Zuweisung von Benutzern zu Servern}

\nl Benutzer k"onnen den VPN-Servern, mit denen Sie sich verbinden k"onnen, auf
verschiedene Weisen zugewiesen werden.

\begin{items}
	\item Per Drag'n'Drop im ``Server/Benutzer''-Reiter
	\item "Uber die Seitenleiste der Detail-Ansicht des Benutzers
	\item "Uber die Seitenleiste der Detail-Ansicht des Servers
\end{items}


\subpoint{Synchronisierung der Benutzer- und Zertifikatsdaten}

\nl Ist ein Server in der Anwendung vorhanden kann mit ihm synchronisiert werden.
Die Synchronisation dient dazu, den Server zu aktualisieren, indem folgende Dateien
auf den Server hochgeladen werden:
\begin{items}
\item Root- und Serverzertifikat
\item Zertifikatssperrliste
\item \Npwf{}
\end{items}
Die  \Npwf{} wird dabei aus den aktuell mit dem Server verbundenen
Benutzern erstellt.

\newpage
\newitem{Netzwerkverwaltung}

\subpoint{Server im Leitstand "offnen}

\nl Durch einen Rechtsklick auf einen Server in der Serverliste gelangt man zu
dem Men"upunkt ``Im Leitstand \"offnen''. Abh"angig davon welche SSH-Authentifizierungsmethode
f"ur diesen Server konfiguriert wurde, wird im n"achsten Schritt nach dem Passwort
f"ur den Login oder nach der Passphrase f"ur den SSH-Key, sofern dieser durch eine
Passphrase gesch"utzt ist, gefragt.\\

\noindent
\begin{center}
 \parbox[b]{84mm}{\textbf{\small
  bytemine empfiehlt die Verwendung von\\
  passwort-gesch"utzten SSH-Schl"usseln.
 }}{\Huge\bfseries!}
\end{center}

\nl War die Authentifizierung erfolgreich, so "offnet sich ein
weiterer Reiter im Hauptfenster des \Nbm.

Zun"achst ist nur der untere Teil des Reiters aktiv. Dort werden die auf dem
Server verf"ugbaren Kan"ale angezeigt -- jeweils mit der M"oglichkeit
den Kanal zu "offnen.
%
Pro OpenVPN-Server-Dienst auf dem jeweiligen Server gibt es einen Kanal. Mittels
des Kanals wird der Zugriff auf die OpenVPN Management-Schnittstelle realisiert.
%
Wurde ein Kanal durch ''Verbinden'' ge"offnet, werden im zugeh"origen Reiter
die verschiedenen Informationen, die "uber den Kanal empfangen
werden, dargestellt.

\subpoint{Mit einem Server verbundene Benutzer anzeigen}

\nl Folgende Informationen
werden pr"asentiert:

\begin{description}
	\item [CN]: Der CommonName auf dem Zertifikat des Benutzers
	\item [IP]: Die IP Adresse von der aus der Benutzer verbunden ist
	\item [VPN IP]: Die IP Adresse die dem Benutzer im VPN zugewiesen wurde
	\item [Empfangen]: Die vom VPN-Server empfangenen Daten des Benutzers
	\item [Gesendet]: Die vom VPN-Server an den Benutzer gesendete Daten
	\item [Verbunden seit]: Zeitpunkt seitdem der Benutzer mit
		dem VPN Server verbunden ist
\end{description}

\nl Handelt es sich bei dem VPN um eine Verbindung im sogenannten ``Bridge''-Modus
so wird statt der \textbf{VPN IP} die \textbf{Virtuelle MAC} der Verbindung angezeigt.

\nl Weiterhin stehen folgende Schaltfl"achen zur Verf"ugung:

\begin{description}
	\item [Status]: Hiermit kann der aktuelle Status abgerufen werden. Dies
	    geschieht auch automatisch alle 30 Sekunden.
	\item [Version]: Hier"uber wird die aktuelle Version des OpenVPN Servers
	    angezeigt.
	\item [Serverlog anzeigen/ausblenden]: Ein Echtzeit-Log kann hiermit an- und ausgeschaltet
	    werden.
	\item [Befehl senden]: Hiermit kann ein Befehl an das OpenVPN
	    Management Interface gesendet werden. \footnote{Die Dokumentation zum
	    OpenVPN Management Interface ist hier einzusehen:
	    \url{http://openvpn.net/index.php/open-source/documentation/miscellaneous/79-management-interface.html}}
        \\ Beispiel: ''verb 5'' zeigt detaillierte Ausgaben im Logging
\end{description}


\subpoint{Mit einem Server verbundene Benutzer trennen}

\nl In der Liste der mit einem Server verbundenen Benutzern, kann durch einen
Rechtsklick auf den jeweiligen Benutzer die Aktion ``Benutzer trennen''
ausgew"ahlt
werden. Hierdurch wird die Verbindung des Benutzers zum VPN-Server getrennt.

\newpage
\newitem{Vorbereitung des OpenVPN Servers}

\subpoint{Verbindung mit einem regul"aren OpenVPN Server}

\nl Der Vorteil der \Nboa{} ist,
dass diese bereits f"ur die Nutzung mit dem \Nbm{} vorbereitet ist.
Jedoch wurde der \Nbm{} auch f"ur die
Nutzung mit regul"aren OpenVPN Servern entworfen.\\\\
Folgende Komponenten m"ussen dazu auf dem OpenVPN-Server installiert
sein:

\begin{items}
	\item Das Software-Paket \textsl{socket-wrapper}
		(wird zusammen mit dem \Nbm{} ausgeliefert).
	\item Das Software-Paket \textsl{bytemine-manager-integration}
		(wird ebenfalls mit dem \Nbm{} ausgeliefert).
	\item OpenVPN in der Version 2.1rc14 oder neuer
\end{items}
\subpoint{Kommunikation mit Unix Domain Sockets}

\nl Seit der Version 2.1rc14 unterst"utzt der OpenVPN Server
zur Kommunikation mit der Management-Schnittstelle Unix-Domain Sockets.
In der Konfigurationsdatei des OpenVPN Servers wird dies wie folgt angegeben:

\nl\code{\ \ \ management /var/run/management-udp unix}

\nl In diesem Falle w"are der Socket dann als
\code{management-udp} im Verzeichnis \code{/var/run} konfiguriert.
%
Auf diesem Unix-Domain Socket wird aus dem Manager "uber
eine SSH-Verbindung zugegriffen.
Damit dies m"oglich ist, existiert das Programm \code{ut} aus dem
Software-Paket \textsl{socket-wrapper}.

Damit der \Nbm{} die Benutzer synchronisieren
kann, m"ussen die entsprechenden Verzeichnisse auf der Serverseite
f"ur den Benutzer, mit dem sich der \Nbm{}
per SSH verbindet, schreibbar sein.

\newpage

\subpoint{Installation des \textsl{socket-wrapper}}

\nl Die Installation des
\textsl{socket-wrapper}-Paketes\footnote{http://www.bytemine.net/de/bytemine-download
} unter Unix-Systemen ist einfach: Nach Entpacken mittels\\
\textit{\code{tar -xzf (socket-wrapper-packet)}}\\
wechelt man in das entpackte Verzeichnis und installiert \textit{ut} mit\\
\textit{\code{make install}}\\

Anschlie"send gibt es das Programm \code{/usr/local/sbin/ut}, sowie die
manual pages \code{ut(8)} und \code{ut.conf(8)}.

\subpoint{Konfiguration des \textsl{socket-wrapper}}

\nl Der Socket-Wrapper wird "uber die Datei \code{/etc/btm/ut.conf}
konfiguriert.  In dieser Datei werden die Unix-Domain Sockets der
einzelnen OpenVPN-Dienste den Kan"alen, auf die der \Nbm{} zugreifen
kann, zugeordnet.

F"ur jeden OpenVPN Dienst wird eine \code{channel}-Anweisung geschrieben.
Hierbei definiert die \code{type}-Anweisung die Art des Kanals (f"ur den
Zugriff auf einen OpenVPN Management Port ist dies \code{VPNM}).
Mittels des \code{method}-Schl"usselwortes wird dann der Weg zu dem
Socket von OpenVPN gewiesen.

\begin{verbatim}
channel "mgmt udp" {
    type VPNM
    method { unix "/var/run/management-udp" }
}
\end{verbatim}
\newpage

\section{Verschiedenes}

\newitem{Probleme mit dem \Nbm ?}
Sollten sie Probleme mit dem \Nbm haben,
schicken Sie bitte eine m"oglichst detaillierte Fehlerbeschreibung
an \textsl{support@bytemine.net}.
Hier ist es hilfreich wenn sie die Versions- und Build-Nummer des von ihnen
verwendeten \Nbm s angeben; sie finden diese Informationen in der Anwendung unter dem Men"upunkt
``Men\"u$\,\rightarrow\,$Info''.

\newitem{OpenVPN-Benutzerkonfigurationserstellung}
Um das Verwalten von Benutzern zu erleichtern, erstellt der \Nbm{} automatisch
 f"ur jeden Benutzer OpenVPN-Konfigurationsdateien. Diese Konfigurationen
liegen jedem Benutzer im Exportverzeichnis bei.\\
Generell erstellt der \Nbm{}  f"ur jeden Server, mit welchem der User
verbunden ist, eine Konfiguration. Dabei verwendet er ein Template, auf
welche alle Konfigurationen basieren. Das Template liegt im
Projektverzeichniss unter ``templates'', und kann den eigenen Bed"urfnissen
angepasst werden.

\newitem{OpenVPN-Benutzerkonfiguration}
Es ist mit dem \Nbm{} auch m"oglich einzelnen OpenVPN-Benutzern statische
IP-Adressen zuzuweisen. M"ochte man diese Funktion verwenden, so gen"ugt es in
dem ServerDetails-Tab die Option 'OpenVPN statische IP f"ur Benutzter' zu
aktivieren, sowie folgende Zeile ihrer OpenVPN-Serverkonfigurationsdatei
hinzuzuf"ugen:\\[2mm]
\textit{client-config-dir (Pfad zum cc-Verzeichnis)}\\[2mm]
M"ochte man statische IP-Adressen verwenden, so sollte man darauf achten
nur Adressen aus einem sukzessiven /30 Subnetz zu verwenden, um
kompatibal mit Windows zu sein.
\textit{Dazu muss das letzte Oktet einer IP-Adresse aus folgendem Set ausgewaehlt
werden}
\footnote{\url{http://openvpn.net/index.php/open-source/documentation/howto.html\#policy}}
\begin{verbatim}
[1 ]  [ 5]  [ 9]  [13]  [17]
[21]  [25]  [29]  [33]  [37]
[41]  [45]  [49]  [53]  [57]
[61]  [65]  [69]  [73]  [77]
[81]  [85]  [89]  [93]  [97]
[101] [105] [109] [113] [117]
[121] [125] [129] [133] [137]
[141] [145] [149] [153] [157]
[161] [165] [169] [173] [177]
[181] [185] [189] [193] [197]
[201] [205] [209] [213] [217]
[221] [225] [229] [233] [237]
[241] [245] [249] [253]
\end{verbatim}
Der \Nbm{} setzt hierbei automatisch den Endpunkt (IP-Adresse f"ur
den Server).

%\newitem{Glossar}

