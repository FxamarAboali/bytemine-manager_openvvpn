
\section{Installation}
\newitem{Installing \& Starting}

%
Unzip \Nvma{}. \Nbn{} is the version respectively the build number in
development versions.
%
\nl The zip archive contains:
%
\begin{items}
 \item   \Nvmj{}: main application in JAR archive.
 \item   \Nvmw{}: executable file in Windows and Mac OS X
 \item   \Nvms{}: executable JAR archive.
 \item   \code{db/}: directory where the application database is stored.
 Due to this fact the installation should be stored on a crypted volume.
 \item   \code{lib/}: external program libraries.
\end{items}

%
A Java Runtime Environment version 6 or 7 is necessary for running the
application.
%
%
On Microsoft Windows and Mac OS X (from version 10.5.x) the application can
be started by
doubleclicking \Nvms{}.
%

On a command line the follwoing has to be entered:

\nl{\small\texttt{\ \ \ java -jar\ }\Nvms{}}

\newitem{Getting started}
%
Upon the first start a dialog for configuring the application will appear.
These settings are the foundation for working with the \Nbms.
Besides choosing the application language, the active modules in the
application can be selected.

\nl Following modules are available:

\begin{items}
 \item management of certificates, including user management
 \item control center: An interface for controlling running OpenVPN
       servers
\end{items}

\nl Additionally in this dialog the format of certificates can be
    choosen:
	\code{PKCS\#12} and \code{Base64} encoding are the options.
%
Furthermore there is the option to specify an export directory, where
imported and generated certificates and their keys are stored.

%
There is also the option to generate OpenVPN configuration files for servers
and users. If activated these files would be created on any change made to a 
server or user and get written into the corresponding directory in the file 
system.

\newpage

\newitem{Import of existing data}

\nl If there is already an existing \Nbms with data to reuse the path to this
database can be specified within the configuration dialog. When changing the
database, a restart is required. All changes prior to restart will be discarded.

\nl Once the initial configuration has been saved the user is prompted
whether he wants
to import existing OpenVPN data (Root CA, certificates and users). Later the
importer is
also available from the menu ``File$\,\rightarrow\,$User Import''. Further
it is possible to import data from the filesystem as well as from an
LDAP directory structure.
%

Afterwards there will be a dialog asking the operator if users shall be
created from the CNs\footnote{%
\sffamily ``Common Name'' -- identifiying textcomponent of a certificate} of
the
certificates subjects. User and certificate will automatically get
connected.
%

\nl If an initial import is skipped, the user is prompted with the
configuration of
the certificate authority.

\nl\textbf{Caution: }\begin{quote} Due to export limitations inside the Java
runtime environment it can occur in some cases that a higher key strength
than 1024 Bit will produce errors.  To avoid this you should install new
Java policy files to your system, before you create even one certificate.
\footnote{If you are affected by this problem, for Java 6 and Java 7 the 
"Java Cryptography Extension (JCE) Unlimited Strength Jurisdiction Policy 
Files 6" are available at:
\url{http://www.oracle.com/technetwork/java/javase/downloads/index.html}.
Afterwards these files have to be copied to JAVA\_HOME/lib/security.  The
old files will be automatically overwritten.} \end{quote}

%
\newpage
%

\section{Administration}

\newitem{Within the application}
%
After configuring the application for the first time the following
tabs, depending on which module you have activated, will appear.

\begin{items}
   \item X509: Overview over certificates
   \item Server: Overview and interface for managing the servers
   \item User: Overview and interface for managing the users
   \item Server/User: Alternative, tree-based view on servers and users.
\end{items}

%
%\subpoint{Updates}	- currently not supported
%

\subpoint{Import}

\nl With ``File$\,\rightarrow\,$User import'' the import can be triggered
(multiple times if needed). A dialog will appear, in which the generation of
users from the certificates CNs can be activated.

\newitem{Administration of certificates}

\nl All managed certificates can be viewed from the ``X509'' tab by
selecting the certificate in the overview and double-click the entry. In
the following dialog all details are shown. With an option under
 ``Edit$\rightarrow$Show revoked certificates'' it is possible just to view
 all valid certificates.

\nl To get direct access to the certificate of a user or a server the
detailed view can also be opened from the users or servers details.  By
clicking the ``show'' button in the certificate section in user and server
view the certificates details will appear.


\subpoint{Root certificate}

\nl The application is designed to manage one single root certificate. Every
server
and user certificate is created from this root certificate.

\nl If more than one root certificate shall be managed by the \Nbm{} there
has to
be a separate database for each root certificate. Databases can be switched
easily from the general configuration dialog.

\newitem{Administration of servers}

\nl With ``Edit$\,\rightarrow\,$Create new server`` servers can be added to
the application. In the appearing dialog the following settings can be made:

\begin{items}
	\item Name: Any name can be given
	\item Hostname: Hostname or IP address of the server
	\item Common name: The common name for the server certificate
    \item Organizational unit: The organizational unit for the server certificate
	\item Server type: Choose whether this server is a \Nboa{} or a regular
          OpenVPN server. To connect with a regular OpenVPN server please
          read the chapter ''connect to a regular OpenVPN server''.
	\item Type of authentication: The \Nbm{} communicates with the server
          via the SecureShell (SSH) protocol. Authentication can be done
          via password or with an ssh key.
	\item Username: Username to use for the ssh connection. In case of a
          \Nboa{} this is 'manager'.
	\item Keyfile: Path to the ssh key file
	\item Socket command: The program used for socket communication has to
          be specified with full qualified path. In case of a \Nboa{} the
          default value should be kept.
	\item SSH port: Serverside port on which the ssh-daemon is listening
	\item Valid: Validity of the server certificate
	\item Server path to passwd file: Serverside path to the passwd file,
          if dual factor authentication is used.
	\item Server path to certificates: Server-side path to the directory
	containing the root and server certificate
	\item OpenVPN port: The port the OpenVPN server is listeing
	\item OpenVPN protocol: The protocol the OpenVPN server is using
	\item Network address: 	The network address the server is using
	\item OpenVPN device: The OpenVPN device the server is using
	\item Redirect gateway: All IP network traffic from the client is
	redirected to this server
	\item OpenVPN static IP for users: Assign static IP to users
	\item Server path to configuration directory: Serverside path to the
	client configuration directory. This field is necessary if you want to
	assign static IPs to your clients.
\end{items}

\nl By clicking the plus button further server data can be entered:

\begin{items}
	\item Duplicate-CN: Two connections with same common name are allowed, so
	one certificate can be used by more than one users
	\item User: Reduce the OpenVPN daemon's privileges after initialization
	\item Group: Reduce the OpenVPN daemon's privileges after initialization
	\item Keep alive: The first field states the intervall (in seconds) in
	which the server will ping his clients. The second field is the time
	(in seconds) the server will wait for a response.
\end{items}




%	\label{OpenVpn-Configuration setting}
\nl Another way of adding new servers is through the ``Server/User'' tab. Is
this tab active, the toolbar will show a button labeled ``New server''.


\newitem{Administration of users}

\nl With ``Edit$\,\rightarrow\,$Create new user'' users can be added to the
database. These settings can be made in the following dialog:

\begin{items}
    \item username: Any name can be given to the user
    \item password: Has to be at least six (6) characters long
\end{items}

\nl By clicking the plus button further user data can be entered:

\begin{items}
    \item Common name: The common name for the users certificate
    \item Organizational unit: The organizational unit for the users
certificate
\end{items}

\nl If there are already servers created in \Nbm{} the user can be directly
connected to them during creation.

\nl Another way of adding new users is through the ``Server/User'' tab. Is
this tab active, the toolbar will show a button labeled ``New user''.

\subpoint{Connecting users to servers}

\nl Users can be assigned to OpenVPN servers which they should be able to
connect to in two different ways:

\begin{items}
   \item Via drag'n'drop in the ``Server/User`` tab
   \item Via the sidebar included in the details view of the user
   \item Via the sidebar included in the details view of the server
\end{items}

\subpoint{Synchronisation of user and certificate data}

\nl If there is one or more servers already created it can be sychronised
with the data from the application. A synchronisation will update the server
by uploading the following files to the server:
\begin{items}
\item Root and server certificate
\item Certificate revocation list
\item \Npwf{}
\end{items}
The \Npwf{} contains the users who are currently connected to the server.

\newpage
\newitem{Network managment}


\subpoint{Open server in control center}

\nl By rightclicking a server in the server overview and choosing the entry
``Open in control center'' in the appearing context menue the control center
for the choosen OpenVPN server is opened. Depending on the selected
authentication method
a login password or the passphrase for the chosen ssh key, if secured by a
passphrase, has to be entered.\\

\noindent
\begin{center}
 \parbox[b]{84mm}{\textbf{\small
  bytemine recommends the use of\\
  passphrase secured ssh keys.
 }}{\Huge\bfseries!}
\end{center}

\nl In case of successful authentication a new tab is opened in the main
view of the \Nbm{}

At first only the lower part of the tab is active where the available
channels are displayed with the option to open each channel.
%
For each OpenVPN service on the selected server there is a channel,
providing access to the OpenVPN management interface.
%
After opening a channel with the ''Connect'' button various information
sent via the channel are displayed in the according tab.

\subpoint{Display of connected users}

\nl This information is shown:

\begin{description}
   \item [CN]: Common name from the users certificate
   \item [IP]: The source IP address, from where the user is connected
   \item [VPN IP]: The IP address of the user within the VPN
   \item [Received]: Amount of data the user received from the VPN server
   \item [Sent]: Amount of data the user sent to the VPN server
   \item [Connected since]: Time since the user is connected to the VPN
server
\end{description}

\nl Is the VPN of type 'bridged', the \textbf{Virtual MAC} will be displayed
instead of the \textbf{VPN IP}.

\nl Furthermore there are four buttons to perform the following actions:

\begin{description}
   \item [Status]: Updates the current status. The status will be automatically
updated every 30 seconds.
   \item [Version]: Shows the current OpenVPN server version.
   \item [Show/Hide server log]: For switching on and off a real-time logging.
   \item [Send command]: To send any command to the OpenVPN management interface.
        \footnote{The documentation of the OpenVPN management interface can be found here:
	    \url{http://openvpn.net/index.php/open-source/documentation/miscellaneous/79-management-interface.html}}
        \\ For example: ''verb 5'' sets detailled output verbosity
\end{description}

\subpoint{Disconnect user from a server}

\nl By right-clicking a user from the list of the connected users the action
''disconnect user'' can be chosen. The user will be disconnected from the
server.

\newpage


\newitem{Configure OpenVPN servers}

\subpoint{Connect to a regular OpenVPN server}

\nl One of the advantages of the \Nboa{} is the preparation for
using it with the \Nbm{}. But the \Nbm{} is also designed to work with
regular OpenVPN servers.

Therefor the following components have to be installed on the OpenVPN
server:

\begin{items}
    \item The software package \textsl{socket-wrapper}
        (is shipped with the \Nbm{})
    \item The software package \textsl{bytemine-manager-integration}
        (also shipped with the \Nbm{})
   \item OpenVPN version 2.1rc14 or newer
\end{items}



\subpoint{Prepare an OpenVPN server}

\nl Since version 2.1rc14 OpenVPN supports Unix domain sockets for
communication with the management interface. Therefor the config file of the
OpenVPN server needs to be configured as following:

\nl\code{\ \ \ management /var/run/management-udp unix}

\nl In this case the socket would be configured as \code{management-udp} in
directory \code{/var/run}.
%
This Unix domain socket is accessed from the manager via an ssh connection.
Therefor we created the application \code{ut} from the software package
\textsl{socket-wrapper}.

For enabling the \Nbm{} to synchronize users the corresponding directories
on server side have to be writable to the user the \Nbm{} connects with.

\newpage


\subpoint{Installation of \textsl{socket-wrapper}}

\nl The installation of the
\textsl{socket-wrapper}\footnote{http://www.bytemine.net/en/bytemine-download}
is quite easy. Extract the package with\\
\textit{\code{tar -xzf (socket-wrapper-package)}}\\
Afterwards switch into the unextracted directory and install \textit{ut}
with \\
\textit{\code{make install}}\\

After the installation there will be the application \code{/usr/local/sbin/ut},
and also the manual pages \code{ut(8)} and \code{ut.conf(8)}.


\subpoint{Configuration of \textsl{socket-wrapper}}

\nl Configuration of the socket wrapper is done in \code{/etc/btm/ut.conf}.
In this file the Unix domain sockets of the single OpenVPN services are linked
to the channels the \Nbm{} has access to.

For each OpenVPN service a \code{channel} command is written. The
\code{type} command defines the type of the channel (to access the OpenVPN
management port this is \code{VPNM}). With the \code{method} keyword the way
to access the socket is defined by OpenVPN.

\begin{verbatim}
channel "mgmt udp" {
    type VPNM
	     method { unix "/var/run/management-udp" }
		  }
		  \end{verbatim}

\newpage

\section{Miscellaneous}

\newitem{Having problems with \Nbm ?}
If you experience some difficulties when using the \Nbms, please send us a
detailled error description to \textsl{support@bytemine.net}.
It would be helpful if you send us the version and build number of the \Nbms
you are using. You can find this information in ``Menu$\,\rightarrow\,$Info''.\\

\newitem{OpenVPN configuration}\label{OpenVPN configuration setting}
The \Nbm{} automatically creates OpenVPN onfigurations for each user. You can find
these configuratons in the export directory.
Generally the \Nbm{} creates for every server, the user is connected with, a
configuration. These configurations are based on a template which you can
find in the project directory under ``templates''.
\newpage

\newitem{OpenVPN client configuration}
It is possible to assign static IP addresses to OpenVPN clients with the
\Nbm{}. In order to use this feature you have to activate it in the
ServerDetails tab and include the following option into your
OpenVPN server configuration file:\\[2mm]
\textit{client-config-dir (path to cc-directory)}\\[2mm]
However, the static IP addresses have to be taken from successive /30 subnets in
order to be compatible with Windows. \textit{``Specifically, the last octet in the IP
address of each endpoint pair must be taken from this
set:''}\footnote{\url{http://openvpn.net/index.php/open-source/documentation/howto.html\#policy}}
\begin{verbatim}
[1 ]  [ 5]  [ 9]  [13]  [17]
[21]  [25]  [29]  [33]  [37]
[41]  [45]  [49]  [53]  [57]
[61]  [65]  [69]  [73]  [77]
[81]  [85]  [89]  [93]  [97]
[101] [105] [109] [113] [117]
[121] [125] [129] [133] [137]
[141] [145] [149] [153] [157]
[161] [165] [169] [173] [177]
[181] [185] [189] [193] [197]
[201] [205] [209] [213] [217]
[221] [225] [229] [233] [237]
[241] [245] [249] [253]
\end{verbatim}
The \Nbm{} automatically sets the end-point (IP address for the server).

